\documentclass[12pt]{article}
\usepackage[utf8]{inputenc}
\usepackage{amsmath}
\title{Aufgabe 2}
\date{}
\begin{document}
\section{Abschnitt (a)}

Allgemeine Formel:

  \begin{align}
    p(w_n|x) &= \frac{p(x|w_n) * \frac{1}{3}}{\sum_{j=1}^3 p(x|w_j) * \frac{1}{3}}
  \end{align}
  
Berechnung fuer $x = (3,3)^t$

  \begin{align}
    p(w_1|(3,3)^t) &= \frac{p((3,3)^t|w_1) * \frac{1}{3}}{\sum_{j=1}^3 p((3|3)^t|w_j) * \frac{1}{3}} \\
    p(w_1|(3,3)^t) &= \frac{0.00001964 * \frac{1}{3}}{0.0001964 * \frac{1}{3} + 0.0029 * \frac{1}{3} + 0.00032254 * \frac{1}{3}} \\
    p(w_1|(3,3)^t) &= \frac{0.0000065471}{0.0011} \\
    p(w_1|(3,3)^t) &= 0.006
  \end{align}
  \begin{align}
    p(w_2|(3,3)^t) &= \frac{p((3,3)^t|w_2) * \frac{1}{3}}{\sum_{j=1}^3 p((3|3)^t|w_j) * \frac{1}{3}} \\
    p(w_2|(3,3)^t) &= \frac{0.0029 * \frac{1}{3}}{0.0001964 * \frac{1}{3} + 0.0029 * \frac{1}{3} + 0.00032254 * \frac{1}{3}} \\
    p(w_2|(3,3)^t) &= \frac{0.0029 * \frac{1}{3}}{0.0011} \\
    p(w_2|(3,3)^t) &= 0.8788
  \end{align}
  \begin{align}
    p(w_3|(3,3)^t) &= \frac{p((3,3)^t|w_3) * \frac{1}{3}}{\sum_{j=1}^3 p((3|3)^t|w_j) * \frac{1}{3}} \\
    p(w_3|(3,3)^t) &= \frac{0.00032254 * \frac{1}{3}}{0.0001964 * \frac{1}{3} + 0.0029 * \frac{1}{3} + 0.00032254 * \frac{1}{3}} \\
    p(w_3|(3,3)^t) &= \frac{0.00032254 * \frac{1}{3}}{0.0011} \\
    p(w_3|(3,3)^t) &= 0.0977
  \end{align}

Der Fisch ist also wohl ein $w_2$.

\section{Abschnitt (b)}

  \begin{align}
    p(w_1|(*,0.3)^t) &=\int p(w_1|(x,0.3)^t) dx\\
    p(w_1|(*,0.3)^t) &= 1.8669
  \end{align}
  \begin{align}
    p(w_2|(*,0.3)^t) &=\int p(w_2|(x,0.3)^t) dx\\
    p(w_2|(*,0.3)^t) &= 7.6189
  \end{align}
  \begin{align}
    p(w_3|(*,0.3)^t) &=\int p(w_3|(x,0.3)^t) dx\\
    p(w_3|(*,0.3)^t) &= 10.5142
  \end{align}

Der Fisch wird also als $w_3$ klassifiziert.

\section{Abschnit (c)}

Berechnung fuer $(0.3,*)^t$
  \begin{align}
    p(w_1|(0.3,*)^t) &=\int p(w_1|(0.3,x)^t) dx\\
    p(w_1|(0.3,*)^t) &= 8.5695
  \end{align}
  \begin{align}
    p(w_2|(0.3,*)^t) &=\int p(w_2|(0.3,x)^t) dx\\
    p(w_2|(0.3,*)^t) &= 7.1908
  \end{align}
  \begin{align}
    p(w_3|(0.3,*)^t) &=\int p(w_3|(0.3,x)^t) dx\\
    p(w_3|(0.3,*)^t) &= 4.2397
  \end{align}

Der Fisch wird also also als $w_1$ klassifiziert.
\end{document}
